\documentclass[10pt]{beamer}

\usepackage[francais]{babel}

\usetheme{metropolis}           % Use metropolis theme

\usepackage{listings}

\usepackage{booktabs}
\usepackage[scale=2]{ccicons}

\usepackage{pgfplots}
\usepgfplotslibrary{dateplot}

\title{Automates cellulaires}
\subtitle{Projet de programmation fonctionnelle - PF5}
\date{\today}
\author{Maxime Gourgoulhon, Pierrick Jacquette}
\institute{Université Paris Diderot}
% \titlegraphic{\hfill\includegraphics[height=1.5cm]{logo/logo}}

\begin{document}

\maketitle

\begin{frame}
  \frametitle{Table des matières}
  \setbeamertemplate{section in toc}[sections numbered]
  \tableofcontents[hideallsubsections]
\end{frame}

\section{Introduction}

\begin{frame}[fragile]
	\frametitle{Automate cellulaire}
	
    Un \emph{Automate cellulaire} consiste en une grille régulière de \textbf{cellules} contenant chacune un \textbf{état} choisi parmi un ensemble fini et qui peut évoluer au cours du temps.
    
    L'état d'une cellule au temps $t+1$ est fonction de l'état au temps $t$ d'un nombre fini de cellules appelé son \textbf{voisinage}.
    
    À chaque nouvelle unité de temps, les mêmes règles sont appliquées \alert{simultanément} à toutes les cellules de la grille, produisant une nouvelle \textbf{génération} de cellules dépendant entièrement de la génération précédente.
    
    \hfill \emph{wikipédia}
\end{frame}

\begin{frame}{Voisinage}
\begin{figure}
	\begin{tikzpicture}
		\draw[step=1cm,gray,very thin] (0,0) grid (5,5);
		\fill[TolDarkBlue] (2.1,2.1) rectangle (2.9,2.9);
		\fill[TolDarkBlue] (1.1,2.1) rectangle (1.9,2.9);
		\fill[TolDarkBlue] (2.1,1.1) rectangle (2.9,1.9);
		\fill[TolDarkBlue] (3.1,2.1) rectangle (3.9,2.9);
		\fill[TolDarkBlue] (2.1,3.1) rectangle (2.9,3.9);
	\end{tikzpicture}
	\caption{Von Neumann de rayon 1}
\end{figure}
\end{frame}

\begin{frame}[fragile]
\frametitle{Fichier texte}
\lstset{
  basicstyle=\footnotesize, frame=tb,
  xleftmargin=.2\textwidth, xrightmargin=.2\textwidth
}
\begin{figure}
\caption{Fichier texte de description}
\begin{lstlisting}
7
Regles
AAAAA
AAAAD
AAADA
AAADD
GenerationZero
DAAAAAA
ADAAAAA
AADAAAA
AAADAAA
AAAADAA
AAAAADA
AAAAAAD
\end{lstlisting}
\end{figure}
\end{frame}

\lstset{language=[Objective]Caml}
\begin{frame}[fragile]
	\frametitle{Types}
	\lstset{
	basicstyle=\footnotesize, frame=tb,
	xleftmargin=.2\textwidth, xrightmargin=.2\textwidth
	}
\begin{figure}
\caption{Core.ml}
\begin{lstlisting}
type state = A | D;;
type generation = state array array;;
type automaton = state array;;
\end{lstlisting}
\end{figure}
\end{frame}

\begin{frame}[fragile]
	\frametitle{Règles}
	\lstset{
	basicstyle=\footnotesize, frame=tb,
	xleftmargin=.2\textwidth, xrightmargin=.2\textwidth
	}
\begin{figure}
\caption{Structure des règles (type \texttt{automaton})}
\begin{lstlisting}
00 : AAAAA : 00000 -> A
01 : AAAAD : 00001 -> A
02 : AAADA : 00010 -> A
03 : AAADD : 00011 -> A
04 : AADAA : 00100 -> D
           -
           -
           -
31 : DDDDD : 11111 -> D
\end{lstlisting}
\end{figure}
\end{frame}

\section{Simulation}

\begin{frame}[fragile]
	\frametitle{Simulation}
\begin{figure}
\caption{stable.ml}
\begin{lstlisting}
let next_generation (aut:automaton) (gen:generation) =
    let size = length gen in
    let next_gen = make_matrix size size D in
    for i = 0 to size - 1 do
        for j = 0 to size - 1 do
	        next_gen.(i).(j) <- next i j gen aut
        done;
    done; next_gen
;;
\end{lstlisting}
\end{figure}
\end{frame}

\begin{frame}[fragile]
	\frametitle{Simulation - next}
\begin{figure}
\caption{stable.ml}
\begin{lstlisting}
let next i j (gen:generation) (aut:automaton) =
   if is_rule aut [
      gen.(i).(j);
      gen.(i).(if j>0 then j-1 else length gen -1);
      gen.((i+1) mod (length gen)).(j);
      gen.(i).((j+1) mod (length gen));
      gen.(if i>0 then i-1 else (length gen -1)).(j)
   ] then A else D
;;
\end{lstlisting}
\end{figure}
\end{frame}

\section{Automate -> FNC}

\begin{frame}{Automate -> FNC}
	\begin{itemize}[<+- | alert@+>]
		\item trouver toutes les règles "instables"
		
		ex : $DADDA \rightarrow D$ est instable
		
		 $(\neg x(i-1,\, j) \wedge x(i,\, j+1) \wedge \neg x(i+1,\, j) \wedge \neg x(i,\, j-1) \wedge x(i,\, j))$
		 
		\item négation, toutes les règles "stables"
		
		$( x(i-1,\, j) \vee \neg x(i,\, j+1) \vee x(i+1,\, j) \vee x(i,\, j-1) \vee \neg x(i,\, j))$
		
		\item lier toutes ces disjonctions pour obtenir la formule sous FNC.
		
	\end{itemize}
\end{frame}


\section{MiniSat}

\begin{frame}[fragile]
\begin{figure}
\caption{fnc.dimacs}
\begin{lstlisting}
p cnf 25 25
20 21 5 24 -25 20 21 5 -24 -25 ... -20 -21 5 -24 25 0
19 25 4 23 -24 19 25 4 -23 -24 ... -19 -25 4 -23 24 0
18 24 3 22 -23 18 24 3 -22 -23 ... -18 -24 3 -22 23 0
 ...
22 3 7 1 -2 22 3 7 -1 -2 ... -22 -3 7 -1 2 0
21 2 6 5 -1 21 2 6 -5 -1 ... -21 -2 6 -5 1 0
\end{lstlisting}
\end{figure}
\end{frame}

\begin{frame}[fragile]
\lstset{basicstyle=\footnotesize}
\begin{figure}
\caption{fnc.dimacs}
\begin{lstlisting}
let rec show_stable () =
    let _ = command "./minisat fnc.dimacs tmp_gen >> log" in
    let gen = is_stable "tmp_gen" in
    if gen = "" then
        print_string "\ndone\n"
    else
    begin
        add_line (inv (split (regexp " +") gen)) "fnc.dimacs";
        add_line gen "gens";
        show_stable ()
    end
;;
\end{lstlisting}
\end{figure}
\end{frame}

\plain{Questions?}

\end{document}
